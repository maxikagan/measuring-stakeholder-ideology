\section{Methodological Appendix: Validation Against Twitter-Based Measures}
\label{sec:appendix_validation}

\subsection{Overview}

This appendix validates our foot traffic-based consumer partisan lean measures against an alternative approach: Twitter-based brand ideology scores from \citet{schoenmueller2023polarized}. The Schoenmueller et al. measure infers brand political positioning from the partisan composition of brands' Twitter followers, providing an independent benchmark derived from entirely different data and methodology.

We compare the two approaches at the brand level, examining correlation, concordance, and systematic differences. The goal is to assess convergent validity: if both measures capture the same underlying construct (consumer political composition), they should be positively correlated despite methodological differences.

\subsection{The Schoenmueller et al. Measure}

\citet{schoenmueller2023polarized} develop a measure of brand political orientation by analyzing the Twitter follower networks of major consumer brands. Their approach proceeds as follows:

\begin{enumerate}
    \item \textbf{Follower Collection}: For each brand's official Twitter account, they collect a sample of followers.

    \item \textbf{Political Classification}: Each follower is classified as liberal or conservative based on whether they follow predominantly liberal or conservative political accounts (politicians, pundits, media outlets).

    \item \textbf{Brand Score Construction}: The brand's political orientation is computed as the share of followers classified as conservative, yielding a score from 0 (entirely liberal followers) to 1 (entirely conservative followers).
\end{enumerate}

The resulting dataset covers 1,289 brands across multiple consumer categories, including retail, restaurants, consumer packaged goods, and services. The measure has been validated against consumer surveys and predicts partisan differences in brand attitudes and purchase intentions.

\subsection{Linking the Datasets}

We link Schoenmueller et al. brand scores to our Advan-based measures using brand name matching. The linking process involves:

\begin{enumerate}
    \item \textbf{Semantic Embedding}: We embed both Schoenmueller and Advan brand names using OpenAI's \texttt{text-embedding-3-large} model (1536 dimensions), then identify candidate matches via cosine similarity.

    \item \textbf{String Distance}: We compute Jaro-Winkler similarity for all candidate pairs, which effectively handles common brand name variations (e.g., ``McDonalds'' vs. ``McDonald's'', ``7eleven'' vs. ``7-Eleven'').

    \item \textbf{Candidate Selection}: We retain pairs with either (a) Jaro-Winkler $\geq$ 0.85, or (b) cosine similarity $\geq$ 0.85 and rank 1, yielding 1,036 candidate pairs.

    \item \textbf{Manual Verification}: Each candidate pair was manually classified as a true match or false positive, accounting for brand variants (e.g., ``Foot Locker'' and ``Kids Foot Locker'' both match to ``Footlocker'').
\end{enumerate}

\begin{table}[htbp]
\centering
\caption{Brand Matching: Schoenmueller et al. to Advan}
\label{tab:brand_matching}
\begin{threeparttable}
\begin{tabular}{lc}
\toprule
& Count \\
\midrule
Schoenmueller et al. brands & 1,289 \\
Candidate pairs generated & 1,036 \\
Manually verified as TRUE match & 662 \\
Manually verified as FALSE match & 374 \\
\midrule
Unique Schoenmueller brands matched & 416 \\
Final validation sample (with foot traffic data) & \textbf{283} \\
\bottomrule
\end{tabular}
\begin{tablenotes}[flushleft]
\small
\item \textit{Notes}: Matching between Schoenmueller et al. (2023) Twitter-based brand scores and Advan branded POI data. Candidate pairs selected via cosine similarity of semantic embeddings and Jaro-Winkler string distance. Each pair manually verified. Final sample excludes brands without foot traffic-based partisan lean estimates.
\end{tablenotes}
\end{threeparttable}
\end{table}

\subsection{Correlation Analysis}

Table~\ref{tab:validation_correlation} presents correlation coefficients between our foot traffic-based consumer partisan lean and the Schoenmueller et al. Twitter-based measure.

\begin{table}[htbp]
\centering
\caption{Correlation: Foot Traffic vs. Twitter-Based Consumer Partisan Lean}
\label{tab:validation_correlation}
\begin{threeparttable}
\begin{tabular}{lccc}
\toprule
Sample & N & Pearson $r$ & Spearman $\rho$ \\
\midrule
All matched brands & 283 & 0.27*** & 0.40*** \\
\midrule
\textit{By Brand Geographic Coverage} & & & \\
National (31+ states) & 120 & 0.32*** & 0.44*** \\
Regional (6--30 states) & 113 & 0.21** & 0.33*** \\
Local (1--5 states) & 25 & 0.18 & 0.25 \\
\bottomrule
\end{tabular}
\begin{tablenotes}[flushleft]
\small
\item \textit{Notes}: Correlations between foot traffic-based visitor Republican lean (our measure, 2020 election data) and Twitter follower-based conservative share (Schoenmueller et al. 2023). Pearson $r$ is linear correlation; Spearman $\rho$ is rank correlation. *** $p < 0.001$, ** $p < 0.01$, * $p < 0.05$. National brands show stronger correlations, consistent with larger sample sizes and more precise estimates.
\end{tablenotes}
\end{threeparttable}
\end{table}

\subsection{Scatter Plot Visualization}

Figure~\ref{fig:validation_scatter} presents a scatter plot comparing the two measures at the brand level.

\begin{figure}[htbp]
\centering
\includegraphics[width=0.85\textwidth]{figures/validation_scatter.pdf}
\caption{Validation: Foot Traffic-Based vs. Twitter-Based Consumer Partisan Lean}
\label{fig:validation_scatter}
\begin{minipage}{0.9\textwidth}
\footnotesize
\textit{Notes}: Each observation is a brand matched across both datasets (N = 283). X-axis: Conservative share of Twitter followers from Schoenmueller et al. (2023). Y-axis: Visit-weighted average Republican two-party vote share of visitor home CBGs (our measure). Dashed line indicates 45-degree perfect agreement. Red solid line is OLS best fit (slope = 0.13, SE = 0.03). Point size proportional to log total visits. Pearson $r$ = 0.27, $p < 0.001$.
\end{minipage}
\end{figure}

\subsection{Sources of Divergence}

The two measures should correlate positively but need not agree perfectly. Several factors may cause divergence:

\subsubsection{Construct Differences}

The measures capture related but distinct constructs:

\begin{itemize}
    \item \textbf{Our measure}: Political composition of \textit{actual visitors} to physical locations, weighted by visit volume.

    \item \textbf{Schoenmueller et al.}: Political composition of \textit{Twitter followers} of brand accounts, which reflects social media engagement rather than in-store patronage.
\end{itemize}

Twitter followers may skew younger, more politically engaged, and more urban than the general customer base. Brands with strong social media presence may attract followers who differ from typical in-store customers.

\subsubsection{Geographic Coverage}

Our measure captures the full geographic footprint of each brand, including rural locations where Twitter penetration may be lower. The Schoenmueller et al. measure may disproportionately reflect urban/suburban customers who are more active on Twitter.

For brands with differential penetration across political geographies---e.g., a chain with both urban and rural locations---the two measures may diverge.

\subsubsection{Temporal Differences}

The Schoenmueller et al. data were collected at a specific point in time (circa 2019--2020), while our foot traffic data span 2019--2024. If brand political associations have shifted over time, this could introduce divergence.

\subsubsection{Measurement Error}

Both measures contain error:

\begin{itemize}
    \item Our measure relies on ecological inference from CBG-level voting to individual visitor preferences.

    \item The Schoenmueller et al. measure relies on inferring follower politics from their following patterns, which may misclassify moderate or cross-partisan individuals.
\end{itemize}

Classical measurement error in both measures would attenuate the observed correlation toward zero.

\subsection{Complete Brand Comparison}

The full brand-level comparison table (N = 283) is available in the online supplementary materials. Brands are sorted by number of U.S. locations (descending), prioritizing major national brands where both measures have the most data and highest precision.

% Full comparison table (N=283) to be generated separately and included as supplementary material.

\subsection{Conclusion}

The validation analysis reveals moderate convergent validity between our foot traffic-based measure and the Schoenmueller et al. Twitter-based measure. The positive correlation ($r = 0.27$, $p < 0.001$) indicates that both approaches capture meaningful variation in consumer political composition, despite using entirely different data sources and methodologies. The Spearman rank correlation ($\rho = 0.40$) suggests stronger agreement in relative orderings than absolute values.

The modest linear correlation reflects systematic differences between the measures rather than noise. Twitter followers self-select based on brand affinity and political identity, while physical visitors are constrained by geography and practical needs. Brands with extreme Twitter followings (e.g., Trump properties with 92\% conservative followers) show the largest divergence from foot traffic patterns, consistent with ideological self-selection on social media. Indeed, the correlation between a brand's Twitter-based extremity and divergence from foot traffic is $r = 0.72$, indicating that Twitter captures performative political consumption while foot traffic captures routine commercial behavior.

The measures are not interchangeable. Each has strengths and limitations: our measure captures actual physical visits across the full geographic footprint; the Twitter measure may better capture social media-engaged consumers and brand perception. The choice between measures should depend on the research question.

For our purposes---examining stakeholder pluralism using comprehensive coverage of commercial activity---the foot traffic approach offers advantages: broader coverage (millions of establishments vs. thousands of brands), granular geographic resolution, and linkage to employee data. The validation against an independent external measure increases confidence that our estimates capture meaningful political composition variation.
